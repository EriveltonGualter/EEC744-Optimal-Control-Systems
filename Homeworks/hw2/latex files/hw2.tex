\documentclass{article}
%
% Demo of the mcode package from 
% http://www.mathworks.co.uk/matlabcentral/fileexchange/8015-m-code-latex-package
% Updated 06 Mar 2014
%

\usepackage{graphicx}
\usepackage{wrapfig}
\usepackage{mathtools}
\usepackage{mathrsfs}
\usepackage{enumitem}
\usepackage{pdflscape}
\graphicspath{ {images/} }

% load package with ``framed'' and ``numbered'' option.
\usepackage[framed,numbered,autolinebreaks,useliterate]{mcode}

% something NOT relevant to the usage of the package.
\usepackage{url}
\setlength{\parindent}{0pt}
\setlength{\parskip}{18pt}


% //////////////////////////////////////////////////

\begin{document}

\title{Homework 2 - Optimal Control Systems}
\author{Erivelton Gualter dos Santos, 2703806}
\date{}

\maketitle 

\begin{enumerate}[]
\item \textbf{Suppose that Q is a non-symmetric matrix, S is the symmetric part of Q, and x is an arbitrary vector. Prove the following:}
$$ x^T Q x = x^T S x $$
\end{enumerate}

As the symetric part of Q is : $ S = \frac{1}{2}(Q+Q^T) $, So:
\begin{eqnarray*}
\begin{split}
	x^T S x &= x^T \frac{1}{2}(Q+Q^T) x \\
	&= x^T \frac{1}{2}(Q+Q^T) x \\
	&= \frac{1}{2}(x^TQ+x^TQ^T) x \\
	&= \frac{1}{2}(x^TQx+x^TQ^Tx) \\
\end{split}
\end{eqnarray*}
Knowing that: $x^TQx=x^TQ^Tx$, So $x^T S x $ = $ \frac{1}{2}(x^TQx+x^TQx)=x^TQx $.

It proves that: $ x^T Q x = x^T S x $

\begin{enumerate}[]
\item[3.] \textbf{Prove the four properties in the left column of Table 1-1 in Kirk's book}
\end{enumerate}

To prove the properties of the linear system State Transition Matrix (STM) it is necessary to use the following representation of STM:
$$ \varphi(t) = e^{At}$$
\
\begin{enumerate}[]
\item[\textbf{a)}] $ \varphi(0) = I $
\end{enumerate}
\begin{eqnarray*}
\begin{split}
	\varphi(t) &= e^{At} = I+At+\frac{A^2t^2}{2\!}+\frac{A^3t^3}{3\!} + ...\\
	\varphi(0) &= e^{A0} = I+A0+\frac{A^20^2}{2\!}+\frac{A^30^3}{3\!} + ...\\
	&= I
\end{split}
\end{eqnarray*}

\begin{enumerate}[]
\item[\textbf{b)}] $ \varphi(t_2-t_1)\varphi(t_1-t_0) = \varphi(t_2-t_0) $
\end{enumerate}

Knowing that $$ \varphi(t_2-t_1) = e^{A \left( t_2-t_1 \right)} $$
And $$ \varphi(t_1-t_0) = e^{A \left( t_1-t_0\right)} $$

We have: 
\begin{eqnarray*}
\begin{split}
	\varphi(t_2-t_1)\varphi(t_1-t_0) &= e^{A \left( t_2-t_1 \right)}e^{A \left( t_1-t_0 \right)} \\
	&= e^{A \left( t_2-t_1+t_1-t_0 \right)} \\
	&= e^{A \left( t_2-t_0 \right)} \\
	&= \varphi(t_2-t_0)
\end{split}
\end{eqnarray*}

\begin{enumerate}[]
\item[\textbf{c)}] $ \varphi^{-1}(t_2-t_1) = \varphi(t_1-t_2) $
\end{enumerate}
\begin{eqnarray*}
\begin{split}
	\varphi^{-1}(t_2-t_1) &= \left( e^{A \left( t_2-t_1\right)}\right)^{-1} \\	
	&= e^{-A \left( t_2-t_1\right)} \\
	&= e^{A \left( t_1-t_2\right)} \\
	&= \varphi(t_1-t_2)
\end{split}
\end{eqnarray*}

\begin{enumerate}[]
\item[\textbf{d)}] $ \frac{d}{dt}\varphi(t) = A\varphi(t) $
\end{enumerate}
\begin{eqnarray*}
\begin{split}
	\frac{d}{dt}\varphi(t) &= \frac{d}{dt}e^{At}\\
	&= Ae^{At}\\
	&= A\varphi(t)\\
\end{split}
\end{eqnarray*}

\begin{enumerate}[]
\item[4.] \textbf{Find the state transition matrices for the systems of Problem 1-12(e), (f), and (h) in Kirk's book}.
\end{enumerate}

\begin{enumerate}[]
\item[\textbf{e)}] $ \frac{Y(s)}{U(s)} = \frac{5(s+2)}{s(s+1)} $
\end{enumerate}
\begin{eqnarray*}
\begin{split}
	\frac{Y(s)}{X(s)}\frac{X(s)}{U(s)} &= 5(s+2)\frac{1}{s(s+1)} \\
\end{split}
\end{eqnarray*}
For:
\begin{eqnarray*}
	\frac{Y(s)}{X(s)} &= 5s(s+2)\\
\end{eqnarray*}
Then, $ y(t) = 5\ddot{x}(t)+10x(t) $

For:
\begin{eqnarray*}
	\begin{split}
	\frac{X(s)}{U(s)} &= \frac{1}{s^2}\\
	U(s) &= s^2X(s) \\
	\end{split}
\end{eqnarray*}
Then, $ u(t) = \ddot{x}(t)$

Assuming the states variables as $x_1 = x(t)$ and $x_2 = \dot{x}(t)$ and $X = [x_1\:\:\:x_2]^T$ and $\dot{X} = [\dot{x}_1\:\:\:\dot{x}_2]^T$, it results in:
\begin{eqnarray*}
\begin{split}
\dot{X} &= 
\begin{bmatrix}
0 & 1 \\
0 & -1
\end{bmatrix} X + u \\ Y &= 
\begin{bmatrix}
10 & 5
\end{bmatrix} X
\end{split}
\end{eqnarray*}

For $\varphi(t) = e^{At} $ and knowing that $e^{At} =Qe^{At}Q^{-1}$: 

The eigenvalues of A is $\lambda_1 = 0 $ and $\lambda_2 = -1 $  therefore, the eingevectors can be:
\begin{eqnarray*}
Q = 
\begin{bmatrix}
0 & -1 \\
0 & 1 \\
\end{bmatrix}
\end{eqnarray*}
And $ Q^{-1} $ :
\begin{eqnarray*}
Q^{-1} = 
\begin{bmatrix}
1 & 1 \\
0 & 1 \\
\end{bmatrix}
\end{eqnarray*}

Therefore: 

\begin{eqnarray*}
\begin{split}
\varphi(t) &= Qe^{At}Q^{-1} \\ &=
\begin{bmatrix}
0 & -1 \\
0 & 1 \\
\end{bmatrix}
\begin{bmatrix}
1 & 0 \\
0 & e^{-t} \\
\end{bmatrix}
\begin{bmatrix}
1 & 1 \\
0 & 1 \\
\end{bmatrix}\\ &=
\begin{bmatrix}
1 & 1-e^{-t} \\
0 & e^{-t} \\
\end{bmatrix}
\end{split}
\end{eqnarray*}


\begin{enumerate}[]
\item[\textbf{f)}] $ \frac{Y(s)}{U(s)} = \frac{(s+1)(s+2)}{s^2} $
\end{enumerate}
\begin{eqnarray*}
\begin{split}
	\frac{Y(s)}{X(s)}\frac{X(s)}{U(s)} &= (s+1)(s+2)\frac{1}{s^2} \\
\end{split}
\end{eqnarray*}
For:
\begin{eqnarray*}
	\begin{split}
	\frac{Y(s)}{X(s)} &= (s+1)(s+2)\\
	Y(s) &= (s+1)(s+2)X(s) \\
	&= [s^2+3s+2]X(s)
	\end{split}
\end{eqnarray*}
Then, $ y(t) = \ddot{x}(t) + 3\dot{x}(t) + 2x(t) $

For:
\begin{eqnarray*}
	\begin{split}
	\frac{X(s)}{U(s)} &= \frac{1}{s^2} \\
	U(s) &= s^2X(s) \\
	\end{split}
\end{eqnarray*}
Then, $ u(t) = \ddot{x}(t) $

Assuming the states variables as $x_1 = x(t)$ and $x_2 = \dot{x}(t)$ and $X = [x_1\:\:\:x_2]^T$ and $\dot{X} = [\dot{x}_1\:\:\:\dot{x}_2]^T$, it results in:
\begin{eqnarray*}
\begin{split}
\dot{X} &= 
\begin{bmatrix}
0 & 1 \\
0 & 0
\end{bmatrix} X + u \\ Y &= 
\begin{bmatrix}
2 & 3
\end{bmatrix} X
\end{split}
\end{eqnarray*}

For $\varphi(t) = \mathcal{L}^{-1} \left\lbrace \left[sI-A \right]^{-1} \right\rbrace $
\begin{eqnarray*}
\begin{split}
\varphi(t) &= \mathcal{L}^{-1} \left\lbrace 
\begin{bmatrix}
s & 1 \\
0 & s
\end{bmatrix} ^{-1} \right\rbrace \\ &= \mathcal{L}^{-1} \left\lbrace 
\begin{bmatrix}
\frac{1}{s} & \frac{1}{s^2} \\
0 & \frac{1}{s^2}
\end{bmatrix} \right\rbrace
\end{split}
\end{eqnarray*}
Therefore:
$$ \varphi(t) = 
\begin{bmatrix}
1 & t \\
0 & 1
\end{bmatrix}
$$

\begin{enumerate}[]
\item[\textbf{h)}] $ \frac{Y(s)}{U(s)} = \frac{4}{(s+1)(s+2)} $
\end{enumerate}
\begin{eqnarray*}
\begin{split}
	\frac{Y(s)}{X(s)}\frac{X(s)}{U(s)} &= 4\frac{1}{(s+1)(s+2)} \\
\end{split}
\end{eqnarray*}
For:
\begin{eqnarray*}
	\frac{Y(s)}{X(s)} &= 4\\
\end{eqnarray*}
Then, $ y(t) = 4x(t) $

For:
\begin{eqnarray*}
	\begin{split}
	\frac{X(s)}{U(s)} &= \frac{1}{(s+1)(s+2)}\\
	U(s) &= [s^2+3s+2]X(s) \\
	\end{split}
\end{eqnarray*}
Then, $ u(t) = \ddot{x}(t) + 3\dot{x}(t) + 2x(t)$

Assuming the states variables as $x_1 = x(t)$ and $x_2 = \dot{x}(t)$ and $X = [x_1\:\:\:x_2]^T$ and $\dot{X} = [\dot{x}_1\:\:\:\dot{x}_2]^T$, it results in:
\begin{eqnarray*}
\begin{split}
\dot{X} &= 
\begin{bmatrix}
0 & 1 \\
-2 & -3
\end{bmatrix} X + u \\ Y &= 
\begin{bmatrix}
1 & 0
\end{bmatrix} X
\end{split}
\end{eqnarray*}

For $\varphi(t) = \mathcal{L}^{-1} \left\lbrace \left[sI-A \right]^{-1} \right\rbrace $
\begin{eqnarray*}
\begin{split}
\varphi(t) &= \mathcal{L}^{-1} \left\lbrace 
\begin{bmatrix}
s & -1 \\
2 & s+3
\end{bmatrix} ^{-1} \right\rbrace \\ 
&= \mathcal{L}^{-1} \left\lbrace 
\begin{bmatrix}
\frac{s+3}{s^2 + 3s + 2}  & \frac{1}{s^2 + 3s + 2} \\
\frac{-2}{s^2 + 3s + 2} & \frac{s}{s^2 + 3s + 2}
\end{bmatrix}
\right\rbrace \\ &= 
\left[\begin{array}{cc} 2\,{\mathrm{e}}^{-t}-{\mathrm{e}}^{-2\,t} & {\mathrm{e}}^{-t}-{\mathrm{e}}^{-2\,t}\\ 2\,{\mathrm{e}}^{-2\,t}-2\,{\mathrm{e}}^{-t} & 2\,{\mathrm{e}}^{-2\,t}-{\mathrm{e}}^{-t} \end{array}\right]
\end{split}
\end{eqnarray*}

\end{document}
